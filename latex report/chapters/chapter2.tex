Traditional image processing systems involved a multi-step approach. Modern CAD systems use Convolutional Neural Networks (CNN’s). These are end-to-end deep learning systems consisting of a single neural network. These deep learning systems require a large amount of training data to outperform the traditional systems\cite{4}. They are highly effective for computer vision and medical image analysis. CAD systems therefore require a large data set of training images to perform a reliable classification of an input image.\newline
\\
Medical Imaging systems are able to control image quality. For classification systems, this is particularly useful since uniform image quality is necessary for uniform performance throughout the data set. Suspicious regions in the image can be located and selected for classification purposes using shape and texture analysis.\newline
\\
 Building a system that can classify images and master computer vision has been a worldwide project with countless applications from facial recognition to self-driving cars. The use of CNN's in this endeavour required a large open source database for all students, scientists and engineers. This database is called ImageNet.\newline
 \\
ImageNet consists of over 14 million individually labelled images of objects. This is useful in medical imaging research as it is often much easier to train a model to see complex features if it has been trained on previous data. These systems are therefore already capable of interpreting edges, shapes and objects in image data.\newline
\\
ImageNet hosts an annual competition where scientists and engineers around the world can enter their unique classifying systems. GoogLeNet, InceptioNet and ResNet are classifiers that were pre-trained on ImageNet and made commercially available because of their involvement in the ImageNet competition. For work solely focused on medical image classification, there is an open source database of thousands of frontal view chest x-rays called ChestX-ray14. \newline
\\
CAD4TB is a commercially available software used to detect tuberculosis \cite{3}. It was made by Delft Imaging Systems, based in the Netherlands. CAD4TB uses textural abnormality, shape detection and feature extraction. It can be used with support vector machines to improve classification. In Africa, it is used to combat TB in impoverished countries. In South Africa it is also used in prisons \cite{5}. The combination of low cost, quick digital x-rays, machine learning and remote medical expertise allows for a quick diagnosis and immediate treatment.\newline
\\
Thanks to CAD4TB, South African patient information was used in \cite{6} to improve diagnoses using machine learning to generate a CAD score based on a combination of chest x-rays and clinical features. The model built significantly outperformed the individual approaches. This was a positive outcome for future TB screening worldwide.\newline
\\
Diagnostic systems created for classification of lung pathologies may use different models and analyze images for various abnormalities, but the goals are common. The goals are to detect abnormalities early \cite{7}, correctly identify the disease, highlight patients that need immediate attention, inform practitioners when more scans are needed \cite{8} and inform radiologists of uncertainties in the system’s diagnosis\cite{2}.\newline
\\
The central challenge is to build a fully automated system that can analyze large quantities of images. The system must be capable of detecting and locating abnormalities and classifying chest x-ray images accurately even when biological variations are present.
