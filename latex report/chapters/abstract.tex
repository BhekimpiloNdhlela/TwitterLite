\thispagestyle{plain}
\begin{center}
    \Large
    \textbf{Web Development/RW334: 405-found}
    
    \vspace{0.4cm}
    \large
    Teacher: Dr. Brink van der Merwe\\
    
    \vspace{0.4cm}
    \textbf{Bheki 1999999 \\ Klensch Lucas 118181818 \\ David Williams 19869355 \\ Keanu 611515\\ }
    
    \vspace{0.9cm}
    \textbf{Abstract}
    
\end{center}
Soil samples obtained in Ijero, Nigeria were assessed to ascertain the radiological risk due to the activity concentration of primordial radionuclides \ce{^238U}, \ce{^232Th} and \ce{^40K} in the soil. This assessment of low energy gamma radiation was done using a Hyper-Pure Germanium (HPGe) Detector and Palmtop MCA software. The soil samples were crushed, dried and sealed before they were measured in the Environmental Radiation Laboratory at iThemba LABS. The measured activity concentration for  \ce{^238U} ranged from 10.7$\pm$2.8 to 94.9$\pm$9.4 Bq/kg with a mean value of 42.0 Bq/kg, for \ce{^232Th} the activity concentration ranged from 4.8$\pm$0.1 to 109.9$\pm$3.8 Bq/kg with a mean value of 40.4 Bq/kg. \ce{^40K} ranged from 15.597 to 954.73 Bq/kg with a mean value of 519.3 Bq/kg. The mean values for the activity concentration were therefore calculated to be higher than the global averages of 30 Bq/kg, 39 Bq/kg and 400Bq/kg for \ce{^238U}, \ce{^232Th} and \ce{^40K} respectively. Of the 21 soil samples that were assessed, 20 of them had hazard indices within the permitted limits. Only 6 of them were below permissible limits for the indoor dose rates, while all of them were within permissible limits for the outdoor dose rates. For 12 of the soil samples the excess lifetime exposure was within the allowed limit. Finally, the percentage risk for the development of cancer was in line with the global average of 5\% for 8 of the soil samples. However the average percentage risk was 5.43\% so it remained outside of what is considered acceptable. Therefore the radiological hazard due to primordial radionuclides for soil samples obtained from Ijero, Ekiti is significant.
